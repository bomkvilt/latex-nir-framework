% this code must be includded rigth after new document's declaration since
%   - it adds relative includes mechanism 
%   - it adds include guards for \usepackage commands

% >> include required packages
% \note the packages must be added to list of visited packages 
% \sa \usepackage
\usepackage{expl3 }
\usepackage{xparse}
\usepackage{xargs }
\usepackage{etoolbox}

\ExplSyntaxOn
\makeatletter

\typeout{*********************************************************************}
\typeout{**~-~-~-~-~-~-~-~-~-~Framework~was~activated~-~-~-~-~-~-~-~-~-~-~-~**}
\typeout{*********************************************************************}

% >> -----------------------------------------------------------------------------------------------
% >> user interface
% >> -----------------------------------------------------------------------------------------------
% >> -------------| include or use new resources

% >> save old command versions
\let \@@packs@old@include    \include
\let \@@packs@old@input      \input
\let \@@packs@old@usepackage \usepackage

% >> declare new commands


% #1 - path to insert
% \note the path can be
%   - relative './lvl_1/lvl_2/.../lvl_n' (with no '..' nodes)
%   - absolut    'lvl_1/lvl_2/.../lvl_n' (with no '..' nodes)
\RenewDocumentCommand{\include}{ m } 
{ \@@pack@insert:Nx \@@packs@old@include {#1} }


% #1 - path to insert
% \note the path can be
%   - relative './lvl_1/lvl_2/.../lvl_n' (with no '..' nodes)
%   - absolut    'lvl_1/lvl_2/.../lvl_n' (with no '..' nodes)
\RenewDocumentCommand{\input}{ m } 
{ \@@pack@insert:Nx \@@packs@old@input {#1} }


% #1 - package arguments
% #2 - package name
% #3 - package date
\RenewDocumentCommand{\usepackage}{ o m o } 
{ \@@packs@usepackage:xxx {#1} {#2} {#3} }


% >> save new commands
\let \@@packs@new@include    \include
\let \@@packs@new@input      \input
\let \@@packs@new@usepackage \usepackage


% >> --------------| current paths


% >> current paths commands
\def \curpath { \str_use:N \@@packs@curpath }
\def \curbase { \str_use:N \@@packs@curbase }


% >> -------------| callbacks

% This command might be called in a pre load event.
% If it's so, the path will not be includded
% \sa \registerLoadEvent
\NewDocumentCommand{\dontIncludeSelf}{  }
{ \bool_gset_true:N \@@packs@bSkip }


% add callbacks that will be called before and after matched paths are includded
% #1 - regular expression
% #2 - pre  load event || -No Value-
% #3 - post load event || -No Value-
% \note callbacks' names will be saved and evaluated only when the pattern string is matched 
%   a current path. So, if the callback is shadowed the shadowing command will be called
% \note callbacks must be DocumentCommands with a signature '(void *)()'
% \note added once the callbacks cannot be removed
% \sa \dontIncludeSelf
\NewDocumentCommand{\registerLoadEvent}{ m m m }
{ \@@packs@addCallback:nnn {#1} {#2} {#3} }


% add callback that will be executed when a current file will be cloed
% #1 - callback
% \note the callback will be called after post-load callbacks
% \sa \registerLoadEvent
\NewDocumentCommand{\registerOnCloseEvent}{ m }
{ \@@packs@addOnClose:n {#1} }


% >> -----------------------------------------------------------------------------------------------
% >> realisation
% >> -----------------------------------------------------------------------------------------------
% >> ----------------------------------| use package

% list of already used packages
\seq_new:N         \@@packs@usedPacks
\seq_gput_right:Nn \@@packs@usedPacks {expl3 }
\seq_gput_right:Nn \@@packs@usedPacks {xparse}
\seq_gput_right:Nn \@@packs@usedPacks {xargs }
\seq_gput_right:Nn \@@packs@usedPacks {etoolbox}

% level of \usepackage command: new input/include commands must be used on a document level only
\int_new:N \@@packs@usepacklvl


% #1 - package arguments
% #2 - package name
% #3 - package date
\cs_new_protected:Npn \@@packs@usepackage:nnn #1 #2 #3 
{
    \ifinlist {#2} {\@@packs@usedPacks} {}
    {   % false

        % mark as visited
        \seq_gput_right:Nn \@@packs@usedPacks {#2}
        
        % set default include commands
        \@@packs@msg@usepackage:n {#2}
        \int_gincr:N \@@packs@usepacklvl
        \let\include \@@packs@old@include
        \let\input   \@@packs@old@input

        % import the package
        \tl_if_novalue:nTF {#1}
        { \tl_if_novalue:nTF {#3} {\@@packs@old@usepackage    {#2}} {\@@packs@old@usepackage    {#2}[#3]} }
        { \tl_if_novalue:nTF {#3} {\@@packs@old@usepackage[#1]{#2}} {\@@packs@old@usepackage[#1]{#2}[#3]} }

        % recover include commands
        \int_gdecr:N \@@packs@usepacklvl
        \int_compare:nNnT {\@@packs@usepacklvl} = {0}
        {
            \let\include \@@packs@new@include
            \let\input   \@@packs@new@input
        }
    }
}
\cs_generate_variant:Nn \@@packs@usepackage:nnn {xxx}


\msg_new:nnn {packages} {@@packs@usepackage} 
{ Info:~usepackage[>>][\int_use:N \@@packs@usepacklvl]: ~ #1 }

\cs_new_protected:Npn \@@packs@msg@usepackage:n #1
{ \msg_term:nnx {packages} {@@packs@usepackage} {#1} }


% >> ----------------------------------| input | include


\cs_new_protected:Npn \@@pack@insert:Nn #1 #2
{
    % >> open new include frame
    \@@packs@frame@start:n {#2}

    \@@packs@msg@input:nn {>>}{\curpath}

    % set a skip variable
    % \note This variable is global so it can be shadowed by nested \@@pack@insert calls.
    %       So, it can be valid only before include-callback (#1) call
    % \note This variable can be overriten with callbacks
    % \sa   \@@packs@clb@call:N, \registerLoadEvent
    \bool_gset_false:N \@@packs@bSkip
    
    % >> call pre-include events
    \@@packs@clb@call:N {preInclude} 
    
    % include
    \bool_if:nTF \@@packs@bSkip
    { \@@packs@msg@input:nn {NO}{\curpath} }
    { #1{\curpath} }

    % call post-include events
    \@@packs@clb@call:N {postInclude}
    \@@packs@onClose

    % stop include frame
    \@@packs@frame@stop:
}
\cs_generate_variant:Nn \@@pack@insert:Nn {Nx}


\msg_new:nnn {packages} {@@packs@input} 
{ Info:~include[#1]: ~ #2 }

\cs_new_protected:Npn \@@packs@msg@input:nn #1 #2
{ \msg_term:nnxx {packages} {@@packs@input} {#1} {#2} }


% >> ------------------| input: frames


\seq_clear_new:N \@@packs@stack@fullpath
\seq_clear_new:N \@@packs@stack@basepath
\seq_clear_new:N \@@packs@stack@onClose

\str_clear:N \@@packs@curpath
\str_clear:N \@@packs@curbase
\tl_clear:N  \@@packs@onClose


% create a new include frame
% #1 - include path
\cs_new_protected:Npn \@@packs@frame@start:n #1
{{
    % split path on elements 
    \seq_clear_new:N   \@@@parts
    \seq_set_split:Nnn \@@@parts {/}{#1}

    \tl_clear_new:N  \@@@firstPart
    \seq_get_left:NN \@@@parts \@@@firstPart
    
    % check if the path is relative
    \tl_if_eq:NnT \@@@firstPart {.} 
    {
        \seq_pop_left:NN \@@@parts \@@@firstPart
        \seq_put_left:Nx \@@@parts {\curbase}
    }

    % remove non-meanning path's elemets
    \@@packs@frame@fixPath:N \@@@parts

    % generate basedir's path
    \seq_clear_new:N  \@@@baseParts
    \seq_set_eq:NN    \@@@baseParts \@@@parts
    \seq_pop_right:NN \@@@baseParts \@@@firstPart % -> dev/null

    % append path stacks
    \seq_gput_right:Nx \@@packs@stack@fullpath {\seq_use:Nnnn \@@@parts     {/}{/}{/}}
    \seq_gput_right:Nx \@@packs@stack@basepath {\seq_use:Nnnn \@@@baseParts {/}{/}{/}}
    \seq_gput_right:Nx \@@packs@stack@onClose  { } % by default - do nothing
}
    \@@packs@actualizeEnv:
}


% close a current include frame
\cs_new_protected:Npn \@@packs@frame@stop:
{{
    \tl_clear_new:N \@@@buffer
    \seq_gpop_right:NN \@@packs@stack@fullpath \@@@buffer
    \seq_gpop_right:NN \@@packs@stack@basepath \@@@buffer
    \seq_gpop_right:NN \@@packs@stack@onClose  \@@@buffer
}
    \@@packs@actualizeEnv:
}


\cs_new_protected:Npn \@@packs@actualizeEnv: 
{
    \@@packs@actualizeEnv@str:NN \@@packs@stack@fullpath \@@packs@curpath
    \@@packs@actualizeEnv@str:NN \@@packs@stack@basepath \@@packs@curbase
    \@@packs@actualizeEnv@tl:NN  \@@packs@stack@onClose  \@@packs@onClose
}


% set global variable from the top of the stack
% #1 - stack
% #2 - variable
\cs_new_protected:Npn \@@packs@actualizeEnv@str:NN #1 #2
{{
    \tl_clear_new:N \@@@buffer
    \seq_get_right:NN #1 \@@@buffer
    \str_gset:NV      #2 \@@@buffer
}}
\cs_new_protected:Npn \@@packs@actualizeEnv@tl:NN #1 #2
{{
    \tl_clear_new:N \@@@buffer
    \seq_get_right:NN #1 \@@@buffer
    \tl_gset:NV       #2 \@@@buffer
}}


\cs_new_protected:Npn \@@packs@frame@fixPath:N #1
{
    \seq_remove_all:Nn #1 {}  % remove 'same directory' parts
    \seq_remove_all:Nn #1 {.} % remove 'same directory' parts
}


% >> ------------------| input: callbacks


\seq_clear_new:N \@@packs@clb@preInclude
\seq_clear_new:N \@@packs@clb@postInclude


% save the spicified callbacks
% #1 - path pattern (fullpath)
% #2 - pre-include  callback
% #3 - post-include callback
\cs_new_protected:Npn \@@packs@addCallback:nnn #1 #2 #3
{{
    \@@packs@msg@newEvent:nnn {#1}{#2}{#3}

    \bool_if:nT {! \tl_if_novalue_p:n {#2}}
    { \seq_gput_right:Nn \@@packs@clb@preInclude {#1::#2} }

    \bool_if:nT {! \tl_if_novalue_p:n {#3}}
    { \seq_gput_left:Nn \@@packs@clb@postInclude {#1::#3} }
}}


% save a current callback to list of callbacks that must be called when a 
% current file will be closed
% \note must be called after \@@packs@clb@postInclude
% \sa \@@packs@stack@onClose, \@@packs@onClose
\cs_new_protected:Npn \@@packs@addOnClose:n #1
{{
    \tl_put_right:Nn   \@@packs@onClose {#1}
    \seq_gpop_right:NN \@@packs@stack@onClose \@@@buffer
    \seq_gput_right:NV \@@packs@stack@onClose \@@packs@onClose
}}


% call callbacks from a callback list with a specified name
% #1 - callback-list's name
\cs_new_protected:Npn \@@packs@clb@call:N #1
{{
    \tl_clear_new:N \@@@arga
    \tl_clear_new:N \@@@argb
    \seq_clear_new:N \@@@parts
    \seq_map_inline:cn {@@packs@clb@#1}
    {
        \regex_split:nnNTF {::} {##1} \@@@parts
        {
            \seq_pop_left:NN \@@@parts \@@@arga
            \seq_pop_left:NN \@@@parts \@@@argb
            \@@packs@clb@call:VVx \@@@arga \@@@argb {\curpath}
        }
        { \@@packs@msg@brokenEvent:nn {##1}{::} }
    }
}}


% call event #2 if patern #1 matches a current path #3
% #1 - path pattern
% #2 - command's name
% #3 - current path
\cs_new_protected:Npn \@@packs@clb@call:nnn #1 #2 #3
{{
    \regex_match:nnT {#1}{#3}
    {
        \@@packs@msg@call:nnn {#2}{#1}{#3} 
        #2
    }
}}
\cs_generate_variant:Nn \@@packs@clb@call:nnn {VVx}


\msg_new:nnn {packages} {@@packs@newEvent} 
{ Info:~new~event[>>]: #1 ~ -> ~ [~#2~::~#3~] }

\cs_new_protected:Npn \@@packs@msg@newEvent:nnn #1 #2 #3
{ \msg_term:nnxxx {packages} {@@packs@newEvent} {#1}{#2}{#3} }


\msg_new:nnn {packages} {@@packs@call} 
{ Info:~call~event[#1]: [#2] ~ -> ~ [#3] }

\cs_new_protected:Npn \@@packs@msg@call:nnn #1 #2 #3
{ \msg_term:nnxxx {packages} {@@packs@call} {#1}{#2}{#3} }


\msg_new:nnn {packages} {@@packs@brokenEvent} 
{ Critical:~storred~event~cannot~be~parsed[!!]:~#1~with~a~separator~'#2' }

\cs_new_protected:Npn \@@packs@msg@brokenEvent:nn #1 #2
{ \msg_fatal:nnxx {packages} {@@packs@brokenEvent} {#1}{#2} }


% >> -----------------------------------------------------------------------------------------------
% >> end of code
% >> -----------------------------------------------------------------------------------------------

\typeout{*********************************************************************}

\makeatother
\ExplSyntaxOff
