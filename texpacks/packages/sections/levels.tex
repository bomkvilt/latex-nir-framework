\texcode{on}

\usepackage{titlesec}
\usepackage{hyperref}
\usepackage{tocloft}

% --------------------------------------------------------------------------------------------------
% >> create section level styles
% --------------------------------------------------------------------------------------------------

% create section classes
\titleclass{\sectionA}{straight}[\section ] % 1
\titleclass{\sectionB}{straight}[\sectionA] % 2
\titleclass{\sectionC}{straight}[\sectionB] % 3
\titleclass{\sectionD}{straight}[\sectionC] % 4
\titleclass{\sectionE}{straight}[\sectionD] % 5
\titleclass{\sectionF}{straight}[\sectionE] % 6

% create section counters: { counter[lvl] }[ counter[lvl - 1] ]
\newcounter{sectionA}           % 1
\newcounter{sectionB}[sectionA] % 2
\newcounter{sectionC}[sectionB] % 3
\newcounter{sectionD}[sectionC] % 4
\newcounter{sectionE}[sectionD] % 5
\newcounter{sectionF}[sectionE] % 6

% create section labels: label[level] = label[lvl - 1] + '.' + counter[lvl]
\renewcommand\thesectionA{             \arabic{sectionA}} % 1
\renewcommand\thesectionB{\thesectionA.\arabic{sectionB}} % 2
\renewcommand\thesectionC{\thesectionB.\arabic{sectionC}} % 3
\renewcommand\thesectionD{\thesectionC.\arabic{sectionD}} % 4
\renewcommand\thesectionE{\thesectionD.\arabic{sectionE}} % 5
\renewcommand\thesectionF{\thesectionE.\arabic{sectionF}} % 6
% fix paragraphs' levels
\renewcommand\paragraph   {\@startsection{paragraph}   {7}{\z@}       {3.25ex \@plus1ex \@minus .2ex}{-1em}{\normalfont\normalsize\bfseries}}
\renewcommand\subparagraph{\@startsection{subparagraph}{8}{\parindent}{3.25ex \@plus1ex \@minus .2ex}{-1em}{\normalfont\normalsize\bfseries}}

% set section title formats and spacing
\titleformat  {\sectionA}{\normalfont\LARGE     \bfseries}{\thesectionA}{1em}{} % 1
\titleformat  {\sectionB}{\normalfont\Large     \bfseries}{\thesectionB}{1em}{} % 2
\titleformat  {\sectionC}{\normalfont\large     \bfseries}{\thesectionC}{1em}{} % 3
\titleformat  {\sectionD}{\normalfont\normalsize\bfseries}{\thesectionD}{1em}{} % 4
\titleformat  {\sectionE}{\normalfont\normalsize\bfseries}{\thesectionE}{1em}{} % 5
\titleformat  {\sectionF}{\normalfont\normalsize\bfseries}{\thesectionF}{1em}{} % 6
\titlespacing*{\sectionA}{0pt}{3.25ex plus 1ex minus .2ex}{1.5ex plus .2ex} % 1
\titlespacing*{\sectionB}{0pt}{3.25ex plus 1ex minus .2ex}{1.5ex plus .2ex} % 2
\titlespacing*{\sectionC}{0pt}{3.25ex plus 1ex minus .2ex}{1.5ex plus .2ex} % 3
\titlespacing*{\sectionD}{0pt}{3.25ex plus 1ex minus .2ex}{1.5ex plus .2ex} % 4
\titlespacing*{\sectionE}{0pt}{3.25ex plus 1ex minus .2ex}{1.5ex plus .2ex} % 5
\titlespacing*{\sectionF}{0pt}{3.25ex plus 1ex minus .2ex}{1.5ex plus .2ex} % 6

% set token levels
\def\toclevel@sectionA{1}
\def\toclevel@sectionB{2}
\def\toclevel@sectionC{3}
\def\toclevel@sectionD{4}
\def\toclevel@sectionE{5}
\def\toclevel@sectionF{6}
\def\toclevel@paragraph{7}
\def\toclevel@subparagraph{8}

% set section numbering settings
\setcounter{secnumdepth}{6} % number of numbered levels
\setcounter{tocdepth}   {3} % number of levels in a table of contents

% table of contents: indenton settings
% |    1.1 text
% |    |<->|    is a #3
% |<-->|        is a #2
% #1 - level specificator
% #2 - row left indent
% #3 - raw-relative text indent
\def\l@sectionA    {\@dottedtocline{1}{0em}{2em}}
\def\l@sectionB    {\@dottedtocline{2}{0em}{2em}}
\def\l@sectionC    {\@dottedtocline{3}{0em}{3em}}
\def\l@sectionD    {\@dottedtocline{4}{0em}{4em}}
\def\l@sectionE    {\@dottedtocline{5}{0em}{5em}}
\def\l@sectionF    {\@dottedtocline{6}{0em}{6em}}
\def\l@paragraph   {\@dottedtocline{7}{0em}{7em}}
\def\l@subparagraph{\@dottedtocline{8}{0em}{7em}}


% --------------------------------------------------------------------------------------------------
% >> create a generic section command
% --------------------------------------------------------------------------------------------------

% add a level specification field
% #1 - temporary inavaliable
% #2 - section level. By default = 1
% #3 - temporary inavaliable
% #4 - section name
\NewDocumentCommand{\newsection}{s R<>{1} o m}
{
    \begin{switch}{#2}
        \case{1}{\sectionA{#4}}
        \case{2}{\sectionB{#4}}
        \case{3}{\sectionC{#4}}
        \case{4}{\sectionD{#4}}
        \case{5}{\sectionE{#4}}
        \case{6}{\sectionF{#4}}

        % \todo raise an exception
    \end{switch}
}

\texcode{}
