\texcode{on}

% >> -----------------------------------------------------------------------------------------------
% >> init section environment
% 
% This document contains scripts that allow to create uniform tables along all the document with
% shorted number of command
% >> -----------------------------------------------------------------------------------------------

% -------------------------------------

% table font size
\let\TableFontSize\small

% min cell
\def\MinRowHeigth{8mm}
\def\TableStretch{1}

% -------------------------------------

% create a new resource type
% #1 - resource group
% #2 - label prefix
% #3 - subpath
\@@secs@res@newResourceType{tabs}{tabs:}{/tabs}


% #1 - output valiable: tl
% #2 - caption function
% #3 - caption
% #4 - level's letter
% #5 - b use global label names
% #6 - short label's name
\cs_new_protected:Nn \@@secs@tabs@GenerateCaption:Nnnnnn
{
    \bool_if:nT {
        ! \tl_if_blank_p:n{#3} &&
        ! \tl_if_blank_p:n{#4} &&
        ! \tl_if_blank_p:n{#6}
    } {
        \tl_clear_new:N \@@@label
        \@@secs@res@makeLabel:Nnnnn \@@@label {tabs}{#4}{#5}{#6}
        \@@secs@tabs@GenerateCaption:NnnV #1 {#2}{#3}\@@@label
    }
}
\cs_new_protected:Nn \@@secs@tabs@GenerateCaption:Nnnn
{
    \tl_set:Nn #1 {#2{#3} \label{#4}}
}
\cs_generate_variant:Nn \@@secs@tabs@GenerateCaption:Nnnn {NnnV}


\cs_new:Nn \@@secs@tabs@GenerateClumnNumbers: {
    \seq_clear_new:N \@@@values
    \int_step_inline:nnn {1} {\l__tblr_c_int} {
        \seq_put_right:Nn \@@@values {\parbox{\cellwidth}{\centering ##1}}
    }
    \seq_use:Nnnn \@@@values { & }{ & }{ & }
}


\newcommand{\@@secs@tabs@caption}[1]
{
    \captionof{table}{#1}
}


\NewDocumentCommand{\@@sec@tabs@Presetup}{ m } {
    \tl_if_blank:nTF {#1} {
        % default code
        \TableFontSize
        \setstretch{\TableStretch}
    } {
        % passed code
        #1
    }
}


% reference a table
% r #1  - flag that represents a search directory 
%       p - part's    resources
%       s - section's resources
%       g - global resources
% s #2  - b use global label names
% m #3  - equation label
\NewDocumentCommand{\refTab}{ r<> s m }
{
    \@@secs@res@ref:nnnnn {tabs}{#1}{#2}{#3}{\c_novalue_tl}
}


% --------------------------------------------------------------------------------------------------

\NewTblrEnviron{sectablebase}
\SetTblrOuter[sectablebase]{long}
\SetTblrInner[sectablebase] {
    rows = {8mm},
}

\NewTblrEnviron{atomictablebase}
\SetTblrOuter[atomictablebase]{tall}
\SetTblrInner[atomictablebase] {
    rows = {8mm},
}


% create a new table
% \note the tables must be created in a new scope
% d #1 - level's letter
% s #2 - b use global label names
% o #3 - short label's name
% o #4 - caption
% D #5 - code before tabularx
\NewDocumentEnvironment{sectable}{ d<> s O{} O{} D(){} } {
    \tl_clear_new:N \@@@label
    \@@secs@res@makeLabel:Nnnnn \@@@label {tabs}{#1}{#2}{#3}

    \@@sec@tabs@Presetup{#5}
    \sectablebase [
        caption = {#4},
        label   = {\tl_use:N \@@@label},
    ]
} {
    \endsectablebase
    \bigskip
}


% create a new table
% G #1 - column scheme
% d #2 - level's letter
% s #3 - b use global label names
% O #4 - short label's name
% O #5 - caption
% O #6 - floating settings
% D #7 - code before tabularx
\NewDocumentEnvironment{atomictable}{G{} d<> s O{} O{} O{H} D(){}} {
    \tl_clear_new:N \@@@label
    \@@secs@res@makeLabel:Nnnnn \@@@label {tabs}{#2}{#3}{#4}

    \tl_clear_new:N \@@@code
    \tl_set:Nn \@@@code {
        \atomictablebase [
            caption = {#5},
            label   = {\tl_use:N \@@@label},
        ]
    }

    \tl_if_blank:nF {#1} {
        \tl_put_right:Nn \@@@code {
            {#1}
        }
    }

    \table[#6]
        \@@sec@tabs@Presetup{#7}
        \tl_use:N \@@@code
} {
        \endatomictablebase
    \endtable
    \bigskip
}


\NewDocumentCommand{\VCELL}{m} {
    \begin{tabular}{@{}c@{}}
        \rotatebox[origin=c]{90}{#1}
    \end{tabular}
}


\texcode{}
