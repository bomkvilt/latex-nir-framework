\texcode{on}

% >> -----------------------------------------------------------------------------------------------
% >> resource envoronments
% 
% Since we have several types of resources like equations and tables that require simular
%   tools to keep current entities roots, to generate label names and to reference labels
%   it's good to place the logic to one layer that will control the processes
% >> -----------------------------------------------------------------------------------------------


\seq_new:N \@@sections@res@resources


% define a new resource type
% #1 - reource group; must be made of chars only
% #2 - label prefix
% #3 - path postfix
% \note resource is a set of the tables:
%   pathTable       -   level letter -> base path that's associated with the letter
%   labelTable      -   level letter -> letter's postfix must be joined with short label name
%   sublabelsTable  -   full label   -> list of suffixes of labels that are packed to the full label
\NewDocumentCommand{\@@sections@res@newResourceType}{ m m m }
{{
    \seq_if_in:NnT \@@sections@res@resources {#1}
    {
        \msg_error:nnx {@@sections}{generic}{Resource~type~#1~is~already~registered}
    }
    \seq_gpush:Nn \@@sections@res@resources {#1}
    \prop_gclear_new:c {@@@sections@#1@pathTable}
    \prop_gclear_new:c {@@@sections@#1@labelTable}
    \prop_gclear_new:c {@@@sections@#1@sublabelsTable}
    
    % create a label prefix
    \tl_gclear_new:c {@@@sections@#1@labelPrefix}
    \tl_gset:cx      {@@@sections@#1@labelPrefix}{#2}

    % create a path postfix
    \tl_gclear_new:c {@@@sections@#1@pathPostfix}
    \tl_gset:cx      {@@@sections@#1@pathPostfix}{#3}
}}

\msg_new:nnn{@@sections}{generic}{ #1 }



% keys to parse \@@sections@res@frame 's arguments
\keys_define:nn { @@sections@res / frame }
{
    bOpen .multichoice:,
    bOpen / on      .code:n = \bool_set_true:N  \@@@bOpen,
    bOpen / off     .code:n = \bool_set_false:N \@@@bOpen,
    bOpen / unknown .code:n = \msg_error:nnxxx { sections } { unknown-choice }
        { key } { on, off } { \exp_not:n {#1} }
}


% open or close a new frame
%  #1 - frame letter (type)
%  #2 - frame path
%  #3 - state \in {on, off}
\NewDocumentCommand{\@@sections@res@frame}{ r<> m m}
{{
    % init a \@@@bOpen variable
    \keys_set:nn { @@sections@res / frame } { bOpen = #3 }

    
    % open or close frames of all registred resources
    \seq_map_inline:Nn \@@sections@res@resources
    {
        % check if the frame operation can be done
        \bool_set:Nn \@@@bOpened {
            \prop_if_in_p:cn {@@@sections@##1@pathTable}  {#1} ||
            \prop_if_in_p:cn {@@@sections@##1@labelTable} {#1} 
        }
        \bool_xor:nnF {\@@@bOpen}{\@@@bOpened}
        { 
            \msg_info:nnx {@@sections}{generic}{\prop_if_in_p:cn {@@@sections@##1@pathTable}  {#1}}
            \msg_info:nnx {@@sections}{generic}{\prop_if_in_p:cn {@@@sections@##1@labelTable} {#1}}
            \msg_error:nnx {@@sections}{generic}{resource~cannot~be~opend~or~closed~twice:~#1~(##1)} 

        }

        % operate over the frame
        \bool_if:NTF \@@@bOpen
        {   % open a frame
            \prop_gput:cnx {@@@sections@##1@pathTable}  {#1} {#2\tl_use:c{@@@sections@##1@pathPostfix}}
            \prop_gput:cnn {@@@sections@##1@labelTable} {#1} {#2}
            \msg_info:nnx {@@sections}{generic}{resource~opened:~#1~(##1)} 
        }
        {   % close the frame
            \prop_gremove:cn {@@@sections@##1@pathTable}  {#1}
            \prop_gremove:cn {@@@sections@##1@labelTable} {#1}
            \msg_info:nnx {@@sections}{generic}{resource~closed:~#1~(##1)} 
        }
    }
}}


% load specified resource to a variable
% #1:tl - output variable
% #2    - resource group
% #3    - table name
% #4    - table key
\cs_new_protected:Npn \@@sections@res@load:Nnnn #1 #2 #3 #4
{
    \prop_get:cnNF {@@@sections@#2@#3} {#4} #1
    {
        \msg_error:nnx {@@sections}{generic}{resource~cannot~be~found:~#2/#3[#4]} 
    }
}

% create a label
%  #1:tl    - output variable
%  #2       - resource group
%  #3       - level's letter
%  #4       - b use global label names
%  #5       - equatoin's label
\cs_new_protected:Npn \@@sections@res@makeLabel:Nnnnn #1 #2 #3 #4 #5
{
    % \todo should I spend compiler's resources to check resource group?
    \tl_if_novalue:nTF {#5}
    {
        \tl_set:Nx #1 {#5}
    }
    {
        % resource group's tag
        \tl_clear_new:N \@@@tag
        \tl_set:Nx      \@@@rag {\tl_use:c {@@@sections@#2@labelPrefix}}

        % set a label
        \bool_if:nTF { #4 }
        { 
            % create a global label
            \tl_set:Nx #1 {\@@@tag#5} 
        }
        { 
            \tl_clear_new:N           \@@@base
            \@@sections@res@load:Nnnn \@@@base {#2}{labelTable}{#3}
            
            % create a local label
            \tl_set:Nx #1 {\@@@tag#5@\@@@base} 
        }
    }
}

% create csv list of full labels from csv-list of short labels
%  #1:tl    - output variable (csv)
%  #2       - resource group
%  #3       - level's letter
%  #4       - b use global label names
%  #5       - equatoin labels (csv)
\cs_new_protected:Npn \@@sections@res@makeLabels:Nnnnn #1 #2 #3 #4 #5
{
    \seq_clear_new:N \@@@labels
    \tl_clear_new:N  \@@@label
    \clist_map_inline:nn { #5 }
    {
        \@@sections@res@makeLabel:Nnnnn \@@@label {#2}{#3}{#4}{##1}
        \seq_push:NV \@@@labels \@@@label
    }
    \tl_set:Nx #1 {\seq_use:Nnnn \@@@labels {,}{,}{,}}
}

% assign suffix to specified label
% #1 - resource group
% #2 - full label name
% #3 - suffix of a subentity
\cs_new_protected:Npn \@@sections@res@addSublabel:nnn #1 #2 #3
{{
    \tl_clear_new:N \@@@parts
    \prop_get:cnNTF {@@@sections@#1@sublabelsTable}{#2} \@@@parts
    { \tl_set:Nx \@@@parts {\@@@parts,#3} }
    { \tl_set:Nx \@@@parts {#3} }

    \prop_gput:cnx {@@@sections@#1@sublabelsTable}{#2}{\@@@parts}
}}


% >> -----------------------------------------------------------------------------------------------
% >> reference specified entities
% >> -----------------------------------------------------------------------------------------------

% reference specified entities from specified resource group
% #1 - resource group
% #2 - level letter
% #3 - b use global label names
% #4 - short label
% #5 - sublabels:csv || -No Value-
\cs_new_protected:Npn \@@sections@res@ref:nnnnn #1 #2 #3 #4 #5
{{
    \tl_clear_new:N \@@@labels
    \@@sections@res@makeLabels:Nnnnn \@@@labels {#1}{#2}{#3}{#4}
    \@@sections@res@ref:cVn {@@@sections@#1@sublabelsTable} \@@@labels {#5}
}}


% reference specfied entities
%   #1  - reference table || -No Value-
%   #2  - full label
%   #3  - sublabels:csv || -No Value-
\cs_new_protected:Npn \@@sections@res@ref:Nnn #1 #2 #3
{{
    % #3 -> ref{#2--v: v \in #3}
    % #1 -> ref{#2--v: v \in labels{#2}} || ref{#2}
    % else -> ref{#2}
    \bool_if:nTF {! \tl_if_novalue_p:n {#3}}
    { \@@sections@res@ref@list:nn{#2}{#3} }
    { 
        \bool_if:nTF {! \tl_if_novalue_p:n {#1}}
        { \@@sections@res@ref@all:Nn{#1}{#2} }
        { \@@sections@res@ref@raw:n {#2} }
    }
}}
\cs_generate_variant:Nn \@@sections@res@ref:Nnn {NVn, cVn}


% reference specified sublabels
%   #1  - full lable
%   #2  - sublabels:csv
\cs_new_protected:Npn \@@sections@res@ref@list:nn #1 #2
{{
    \seq_clear_new:N \@@@labels
    \clist_map_inline:nn { #2 }
    {
        \seq_push:Nn \@@@labels {#1--##1}
    }
    \cref{\seq_use:Nnnn \@@@labels {,}{,}{,}}
}}
\cs_generate_variant:Nn \@@sections@res@ref@list:nn {nV}


% reference a specified label
%   #1  - full label
\cs_new_protected:Npn \@@sections@res@ref@raw:n #1
{{
    \cref{#1}
}}


% reference all labels of the specified base label
%   #1  - reference table:prop
%   #2  - base label
% \note if the table not contains the label the command will be falled back to a @raw realisation
\cs_new_protected:Npn \@@sections@res@ref@all:Nn #1 #2
{{
    \tl_clear_new:N \@@@labels
    \prop_get:NnNTF #1 {#2} \@@@labels
    { \@@sections@res@ref@list:nV {#2} \@@@labels }
    { \@@sections@res@ref@raw:n {#2} }
}}


\texcode{}
