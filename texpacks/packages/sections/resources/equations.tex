\texcode{on}

% >> -----------------------------------------------------------------------------------------------
% >> init section environment
% >> -----------------------------------------------------------------------------------------------

% create a new resource type
% #1 - resource group
% #2 - label prefix
% #3 - subpath
\@@secs@res@newResourceType{eqs}{eqs:}{/eqns}


% input specified equations
% n #1  - equation's path
% n #2  - equation's label
% n #3  - equation's part to pick up
% n #4  - ignore varexpl section
\cs_new_protected:Npn \@@secs@eqns@input:nnnn #1 #2 #3 #4
{{
    \input{#1}

    \printEquations[#2][#3]

    \bool_if:nF {#4}
    {
        \printVarExplTable{Где}
    }
}}
\cs_generate_variant:Nn \@@secs@eqns@input:nnnn {xxnn, VVnn}


% >> -----------------------------------------------------------------------------------------------
% >> document interface
% >> -----------------------------------------------------------------------------------------------


% include converted .eps equation to co current position
% r #1  - flag that represents a search directory 
%       p - part's    resources
%       s - section's resources
%       g - global resources
% m #2  - equation path
% s #3  - b use global label names
% o #4  - equation label
% d #5  - equation's part to pick up
% s #6  - ignore varexpl section
\NewDocumentCommand{\addEquation}{ r<> m s o d<> s}
{{
    % path to equation
    \tl_clear_new:N \@@@path
    \@@secs@res@load:Nnnn \@@@path {eqs}{pathTable}{#1}
    \tl_set:Nx \@@@path {\@@@path/#2.tex}

    % create a base equation label
    \tl_clear_new:N \@@@label
    \@@secs@res@makeLabel:Nnnnn \@@@label {eqs}{#1}{#3}{#4}

    % import specified subequations with the label
    \@@secs@eqns@input:VVnn \@@@path \@@@label {#5}{#6} 
}}


% register specofied suffix to specified label
% #1 - full label name
% #2 - sublabel
\NewDocumentCommand{\registerEquationPart}{ m m }
{
    \@@secs@res@addSublabel:nnn {eqs}{#1}{#2}
}


% reference equation
% r #1  - flag that represents a search directory 
%       p - part's    resources
%       s - section's resources
%       g - global resources
% s #2  - b use global label names
% m #3  - equation label
% d #4  - equation's parts: csv
\NewDocumentCommand{\refEq}{ r<> s m d<>}
{
    \@@secs@res@ref:nnnnn {eqs}{#1}{#2}{#3}{#4}
}


\texcode{}
