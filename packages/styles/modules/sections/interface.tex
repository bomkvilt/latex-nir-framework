%% ----------------| include sections into a basic peoject
%   def. Section is main logical part of a document is placed in its own directory
%       and have extremely low connectivity with another sections.
% 
%   Each sections have the following structure:
%   /eqs/       - a root directory for MathType .eps/none files
%   /fig/       - a root directory for image (figure) file
%   /math/      - a root directory for Wolfram files
%   /appx.tex   - an entry point describes the section's appendices
%   /main.tex   - an entry point describes the section's body
%   /auto.tex   - an autogenerated file contains all required
%               project structure constants.
% 
%   The sections are placed in a <project_root>/sections/<sections_name> directories.

% includes a sections into a basic project
% @section_name - name of the section placed in a /sections/ directory
\NewDocumentCommand{\IncludeSection}{ m } {
    \secs_sectonBegin:N{#1}

    % check if the section isn't regestred
    \ifinlist{#1}{\g_secs_added_sections_tl} {
        \PackageError{sectoins}{It's not possible to include a #1 section twice}{}
        \stop
    } {}

    % include the section's code
    \subimport{sections/#1/}{main.tex}

    % register the section
    \seq_gpush:Nn \g_secs_added_sections_tl {#1}

    % register an appendix if exists
    \IfFileExists{sections/#1/appx.tex} {
        \seq_gpush:Nn \g_secs_added_appendices_tl {#1}
    } {}
    
    \secs_sectonBegin:N{#1}
}

% include all registered appendices into a basic project
\NewDocumentCommand{\IncludeAppendices}{ o } {
    \seq_map_inline:Nn \g_secs_added_appendices_tl {
        \secs_sectonBegin:N{##1}
        \subimport{sections/##1/}{appx.tex}
        \secs_sectonEnd:N{##1}
    }
}


%% ----------------| include section resources
%   Since all sections resources are placed in known directories
%   it's convenient to create special command to simplify and standardize
%   the resources inclusions.

% Include an equatoin into a section
%   @equation_path      - /eqs - relative equation path
%   ?equation_label     - the equaition's lable
%       \note:  the labale must be the unique for all EQUATIONS in THE section
\NewDocumentCommand{\IncludeEquation}{ m o } {
    \begin{equation}
        \IfValueT{#2} { \label{eqs:\NAMEsection-#2} }
        \input{build/sections/\NAMEsection/\ROOTeqs/#1.tex}
    \end{equation}
}

% Include a figure into a section
%   @figure_path        - /fig - relative figure path
%   ?figure_label       - the figure's lable
%       \note:  the labale must be the unique for all FIGURES in THE section
%   ?caption            - the image's caption
%   ?figure_settings    - settings forwards into a \includegraphics function
\NewDocumentCommand{\IncludeFigure}{ m o o O{ }} {
    \begin{figure}[ht]
        \centering\includegraphics[#4]{#1}
        \IfValueT{#3} { \caption{#3}                }
        \IfValueT{#2} { \label{fig:\NAMEsection-#2} }
    \end{figure}
}

% Include the section's part placed in a separated file
%   @submodule_path     - section_root relative path to a submodule
\NewDocumentCommand{\IncludeSubsection}{ m } {
    \input{#1.tex}
}


%% ----------------| add references

% Create a reference on the section's equation
%   @equation_label - a label was assigned to an equation during an insertion
\NewDocumentCommand{\req}{ m } {
    \secs_addReference:Nn{eqs:\NAMEsection-}{#1}
}

% Create a reference on the section's figure
%   @figure_label   - a label was assigned to a figure during an insertion
\NewDocumentCommand{\rfg}{ m } {
    \secs_addReference:Nn{fig:\NAMEsection-}{#1}
}
